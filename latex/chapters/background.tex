\chapter{Background}
\begin{quotation}
\noindent ``\emph{quote}''
\begin{flushright}\textbf{auteur, date}\end{flushright}
\end{quotation}

\vspace*{0.5cm}

\section{Reinforcement Learning : problem definition}
Reinforcement learning is a field of machine learning inspired by psychology.
It studies how an agent learns to maximize the cumulative reward it receives
from an environment in which it takes given actions. 

\subsection{Markov Decision Processes}
A reinforcement learning problem can be formally defined as a Markov 
Decision Process (MDP) \index{MDP} characterised by :
\begin{itemize}
	\item a set of states $\mathcal{S}$
	\item a set of actions $\mathcal{A}$
	\item a transition function 
		$T(s, a, s') = P(s_{t+1} = s' \mid s_t = s, a_t = a)$
	\item a reward function 
		$r(s, a, s') = \mathbb{E}
		 [r_{t+1} \mid s_t = s, a_t = a, s_{t+1} = s']$
\end{itemize}

The goal of reinforcement learning is for the agent to select, in any state it
can be in, the action that will lead it to the highest expected reward :

\begin{equation}
\mathbb{E}[r] = r_t + \gamma r_{t+1} + \gamma^2 r_{t+2}^2 + ... =
 \sum\limits_{i=0}^\infty \gamma^i r_{t+i}
\end{equation}

\noindent with the discount factor \index{discount factor} $\gamma \in [0, 1[$.
The discount factor allows one to tune the agent's behaviour on the
short-term/long-term spectrum. A discount factor $\gamma=0$ would mean that the
agent maximises its expected reward for the next transition only whereas a
discount factor close to one will favor behaviour that maximises long-term
reward, even if one action leads to a poor reward at first.\\

\subsection{Policy}
The agent uses a policy $\pi(a \mid s)$ which describes a probability
distribution over the action set $\mathcal{A}$, determining the probability of
selecting action $a_i$ from state $s_i$. This policy is 
\textbf{deterministic} if and only if :
\begin{equation}
\forall\, s \in \mathcal{S},\; \exists\, a \in \mathcal{A} : \pi(a \mid s) = 1
\end{equation}
\noindent Otherwise, the policy is \textbf{stochastic}.


\subsection{Policy Iteration}

\subsection{Value Iteration}
V, Q

\subsection{Actor-Critic}


Advantage

