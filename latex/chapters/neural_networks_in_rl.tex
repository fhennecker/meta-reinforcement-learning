\chapter{Neural Networks and Reinforcement learning}
\begin{quotation}
\noindent ``\emph{quote}''
\begin{flushright}\textbf{auteur, date}\end{flushright}
\end{quotation}

\vspace*{0.5cm}


\section{Policy gradient methods}
Policy gradient methods are rather straightforward ways of learning to solve a task.
Training starts with a randomly initialised policy. Let us assume that the
policy is defined by a neural network of which the input layer is set to 
receive an observation of the state of the environment and the output layer
defines a policy (a probability distribution over the actions set).
When the agent performs an action chosen with its policy, the environment will
update its state and the agent will receive an observation of this state as
well as a reward signal. We can use the reward signal to proportionately
encourage (if the reward is 
positive) or discourage (if the reward is negative) taking this action in 
this specific state. \\

Let $R_t$, the discounted future reward at time step $t$, be the metric
that we want to maximise. In a policy gradients method, we will update the
neural network policy directly with :
$$\alpha \nabla \pi(a \mid s) R_t$$
in which $\alpha$ is the learning rate.

\section{Value methods}
There is another category of methods which use an estimate of the value of
actions or states; in other words estimates of their discounted future reward
$R_t$.\\

The optimal state value function 
$$ V^*(s) = \mathbb{E}\left[ r_t + \gamma r_{t+1} + \gamma^2 r_{t+2}^2 + ...  \right]$$
can be estimated using the Bellman equation in the following way :
$$ V(s) = \mathbb{E}\left[ r_t + \gamma V(s')\right]$$
where $s'$ is the state reached after $s$.\\

Indeed, under the condition of optimality, the value of a state is the sum of
the discounted value of the unique optimal next state and the reward of the 
transition to that next state.\\

This can be used as an update rule for the $V$ network with a loss similar to :
$$ \left[V(s) -  \left(r_t + \gamma V(s') \right)\right]^2 $$

\paragraph{V, Q}

\section{Actor-Critic}

\section{The A3C algorithm}
To understand A3C \cite{a3c}, which is one of the most popular reinforcement 
learning algorithms to date, one must first understand the two main groups of
reinforcement learning methods that sometimes intersect. The first is the 
group of policy gradient methods, and the second is the group of value 
methods.




Advantage

